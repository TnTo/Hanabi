\documentclass[]{article}
\usepackage{hyperref}
\usepackage{amsmath}
\usepackage{biblatex}

\bibliography{Hanabi.bib}

\title{Resolve Hanabi with Neural Network}
\author{Ciruzzi Michele}

\begin{document}

\maketitle

\section{Introduction}
Hanabi is a cooperative game for 2 to 5 players with incomplete and asymmetric information.
For this features it has been proposed by DeepMind group \parencite{BARD2020103216} as a new challenge for Reinforcement learning.
The problem has been tackled both with deterministic (e.g \cite{Cox2015}) and ML-learning based algorithm (e.g. \cite{Lerer2019}).
The game has been demostrated to be NP-Hard in general \parencite{Baffier2016}.
As a rule of thumb a good algorithm is able to achieve a perfect score (25 points) in more than the 90\% of games.

\section{Methods}
\subsection{General ideas}
Even if more sofisticated algorithms are avaible, I have implemented the Deep-Q-Learning algorithm introduced by \textcite{Mnih2015}.

\printbibliography{}

\end{document}


